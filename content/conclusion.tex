% !TeX root = ..\main.tex

\chapter{Conclusion}
After dealing with all the different NoSQL databases, this chapter is about
summarizing the most important aspects.

NoSQL is an umbrella term for a variety of different kinds of databases.
Examples of such database systems include key-value stores, document-oriented
databases, wide column databases and graph databases. Historically, SQL
databases were created with the aim of being a “one size fits all” solution for
all applications. Although this might have been realistic at the time of their
inception, nowadays such a solution is infeasible due to the sheer variety of
use cases for a database ranging from storing highly structured data to
completely unstructured data and from low to enormous amounts of data. Most
NoSQL databases on the other hand are designed to be a perfect fit for some use
cases while accepting that this makes them a terrible choice for most other use
cases. Due to that, developers need to choose the right database for their needs
and may need to use multiple different databases in a single application.

From the different databases discussed in this book, some application specific
recommendations are given below: Graph databases are well-suited for databases
with many relationships. CloudFirestore is a fully hosted document database
managed by google. It offers great integration, especially in the world of
mobile development. An open-source alternative would be MongoDB. MongoDB is
another document database and a great choice, if the application has large
amounts of different or changing data models. Use wide column databases like
Cassandra if your system must deal with large amounts of unstructured data. Use
key-value stores if you need a fast data access of data. With regards to the CAP
theorem, we have shown that scalable NoSQL databases cannot forfeit partition
tolerance and are therefore either AP or CP systems.

The book compared the structured approach of SQL databases with the unstructured
approach of NoSQL databases. Furthermore, it named specific advantages and disadvantages
of popular representatives for the different NoSQL databases.
The goal of this paper is to ease the search for a suitable database for given requirements or
specific architectures.

\section{Outlook}
As data continues to grow in importance in our daily lives, the amount of data being
generated is growing exponentially, much of which is unstructured data. This has led to the
need for more flexible databases, such as NoSQL databases. With the rise of big data, more
data is being stored in the cloud, which has further increased the popularity of NoSQL
databases. The fact that many NoSQL databases are open-source projects also makes them
more appealing, as there is a move towards more open-source databases.

Sometimes SQL databases are still preferable, particularly when dealing with highly
structured data. In the future, we can expect continued development and improvement of
NoSQL databases, with efforts to address some of their limitations, such as improving
transaction support and standardization. Additionally, we may see further advancements in
the integration of NoSQL databases with other technologies, such as big data platforms and
cloud computing. As organizations increasingly adopt digital technologies and generate larger
amounts of data, NoSQL databases will continue to play a key role in managing and
processing data efficiently and effectively.

\section{Closing Remarks}
In summary, while NoSQL databases have a common approach to handling big data and
providing flexible schema designs, they differ significantly in their placement on the CAP
theorem, which represents their strengths and weaknesses in terms of consistency, availability,
and partition tolerance. As a result, selecting the right NoSQL database for a particular use
case requires careful evaluation of its unique features and characteristics. However, NoSQL
databases have proven to be a valuable alternative to traditional SQL databases and
organizations can be confident that there is almost always a NoSQL solution that will meet
their data storage needs.
