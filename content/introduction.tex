% !TeX root = ..\main.tex

\chapter{Introduction}
Can relational databases (DBs) keep up with the ever-growing demands of modern
big data architectures? Relational databases are the current state of the art
when it comes to operational data. In the current years, the big data trends
forced to reevaluate database structures. Therefore, NoSQL databases have gained
traction in recent years. This book will deal with different NoSQL databases and
analyze them in detail. It is a result of the course “Current database
technologies” at the Corporate State University Stuttgart. The different
chapters about the databases were written by small groups of students with
limited interaction. Each chapter can be looked at as a small article or paper.
It will explore Cassandra, Redis, Hazelcast, MongoDB, Cloud Firestore, and
Neo4j. In general, the term NoSQL stands for “not only SQL”. It reflects
databases with alternative models outside the standard relational world.

SQL databases have been covered extensively in previous lectures. Principles
such as ACID or BASE have been explained in more detail, and the limitations of
SQL databases have been discussed. Today, more and more applications are based
on NoSQL databases \parencite{solid_it_gmbh_db-engines_2023} and there are many
different topologies of NoSQL databases, such as wide column, document-oriented,
key-value or graph- based databases. As a result, these database topologies need
to be discussed and compared with conventional SQL and other NoSQL topologies. A
theoretical basis for this comparison would be the CAP theorem introduced by
\textcite{brewer_towards_2000}, which deals with the following three different
main characteristics of databases: Consistency, Availability and Partition
Tolerance. The theorem claims that only two of these features can be covered by
one system.

As a result of our motivation and the context of this work, the following
research question was formulated: \enquote{What are the advantages and
disadvantages of current (open source) NoSQL DBs with a theoretical foundation
of the CAP theorem \parencite{brewer_towards_2000} and task/application-oriented
recommendations?}


Exploring databases through sample projects can provide insights into the
strengths and weaknesses of different databases and technologies. However, this
approach has some limitations: We were only able to look at a limited number of
database technologies. The sample of databases and projects chosen may be biased
towards certain types of applications, and the analysis may be limited in depth
due to time and resource constraints. In addition, the landscape of database
technologies is constantly evolving. This may limit the relevance of the
findings to current projects.

In this paper, we will explore six different NoSQL databases, namely Cassandra,
Redis, Hazelcast, MongoDB, The Cloud Firestore, and Neo4j. We will examine the
unique features and benefits of each database, and highlight how they can be
leveraged to meet the needs of various applications. Additionally, we will
explore the trade-offs that come with different consistency, availability, and
partition tolerance levels, as defined by the CAP theorem, and how each database
balances these trade-offs. Through practical examples, readers will gain a clear
understanding of the advantages and limitations of each NoSQL database and how
they can be applied to real-world scenarios. By the end of this paper, readers
will have a deeper understanding of the unique capabilities of each of these
NoSQL databases, and be able to make informed decisions when choosing the right
database for their specific needs.